% -*- coding: utf-8 -*-
\documentclass[11pt]{ltxdoc}

% -- local vars --
\newcommand{\CallName}{International Outgoing Fellowships (IOF)}
\newcommand{\CallID}{FP7-PEOPLE-2011-IOF}
\newcommand{\AppShortTitle}{Application short name}
\newcommand{\AppRef}{MC-IOF-REFNUM}

% -- imports --
\usepackage{geometry}
\geometry{a4paper, hmargin=2cm, vmargin=2cm}
\usepackage{graphicx}
\usepackage{url}
\usepackage{fancyhdr}
\usepackage{lastpage}
\usepackage{tabularx}
\usepackage[table]{xcolor}
%\usepackage{booktabs}
\usepackage{color}
\usepackage{hyperref}
\hypersetup{
  colorlinks=true,
  linkcolor=grey,
  citecolor=black,
  filecolor=magenta,
  urlcolor=blue,
  pdfpagemode=UseThumbs,
  pdfstartview=FitH,
  pdftitle={Your project title},
  pdfauthor={You},
  pdfsubject={Marie Curie Actions Proposal},
  pdfkeywords={Keywords for your project}
}

% -- custom settings --

% -- Page style
\pagestyle{fancy}
\fancyhead{}
\fancyfoot{}
\fancyhead[L]{\AppRef}
\fancyhead[R]{\AppShortTitle}
\fancyfoot[L]{Part B}
\fancyfoot[R]{\upshape Page {\thepage} of \pageref{LastPage}}
\renewcommand{\headrulewidth}{0.5pt}
\renewcommand{\footrulewidth}{0.5pt}

% -- Colors
\definecolor{grey}{gray}{0.2}
\definecolor{darkblue}{rgb}{0.27,0.4,0.87}

% -- Fonts (XeLaTeX)
\usepackage{fontspec} %for loading fonts
\usepackage{xunicode,xltxtra}
\defaultfontfeatures{Mapping=tex-text}
\setmainfont{Ubuntu}
\setsansfont{Ubuntu}
\setmonofont{Ubuntu}

% -- Section numbering
\renewcommand{\thesection}{B\arabic{section}.}
\renewcommand{\thesubsection}{}

% -- No indentation
\setlength{\parindent}{0.0in}
%\setlength{\parskip}{0.1in}

\author{You}
\title{Your project title}
\date{\today}

% -- Formal environment
% Taken from: http://www.jevon.org/wiki/Fancy_Quotation_Boxes_in_Latex
% for adjustwidth environment
\usepackage[strict]{changepage}
% for formal definitions
\usepackage{framed}
% environment derived from framed.sty: see leftbar environment definition
\definecolor{formalshade}{rgb}{0.95,0.95,1}
\newenvironment{formal}{%
  \def\FrameCommand{%
    \hspace{1pt}%
    {\color{darkblue}\vrule width 2pt}%
    {\color{formalshade}\vrule width 4pt}%
    \colorbox{formalshade}%
  }%
  \MakeFramed{\advance\hsize-\width\FrameRestore}%
  \noindent\hspace{-4.55pt}% disable indenting first paragraph
  \begin{adjustwidth}{}{7pt}%
  \vspace{2pt}\vspace{2pt}%
}
{%
  \vspace{2pt}\end{adjustwidth}\endMakeFramed%
}

% -- main purpose --
\begin{document}

% -----------------------------------------------------------------------
% Title page

\thispagestyle{empty}
\begin{center}

\Huge{\bf STARTPAGE}

\vspace{4cm}

\Large{PEOPLE\\MARIE CURIE ACTIONS}

\vspace{4cm}

\Large{\bf \CallName}

\Large{\bf Call: \CallID}

\vspace{4cm}

\huge{PART B}

\vspace{4cm}

\huge{"\AppShortTitle"}

\end{center}

\clearpage

% -----------------------------------------------------------------------
% TOC

\tableofcontents

% -----------------------------------------------------------------------
% Main matter

% --------------------------------------------------------------- Section
\section{Scientific and technological quality (maximum 8 pages)}

\subsection{Research/technological quality, including any
  interdisciplinary and multidisciplinary aspects of the proposal}

\begin{formal}
  Give a clear description of the state-of-the-art of the research
  topic. Provide a clear and specific description of the research
  objectives against the background of the state of the art, and the
  results hoped for. The most relevant bibliographical references
  should be provided, and must be included in the overall page
  count. If relevant, provide information on
  interdisciplinary/multidisciplinary and/or inter-sectoral aspects of
  the proposal.
\end{formal}

\subsection{Appropriateness of research methodology and approach}

\begin{formal}
  For each objective explain the methodological approach that will be
  employed in the project and justify it in relation to the overall
  project objectives. Describe any relevant techniques, methods or
  analyses that will be applied.
\end{formal}

\subsection{Originality and Innovative nature of the project, and
  relationship to the 'state of the art' of research in the field }

\begin{formal}
  Explain the contribution that the project is expected to make to
  advance the state-of-the-art within the project field. Describe any
  novel concepts, approaches or methods that will be employed.
\end{formal}

\subsection{Timeliness and relevance of the project}

\begin{formal}
  Describe the appropriateness of the research proposed against the
  state of the art and why it is timely. Outline the benefit that will
  be gained from undertaking the project at European Research Area
  (ERA) level and how the fellowship will contribute to enhance ERA
  research excellence and reintegrate the researcher. Describe the
  scientific, technological, socio-economic or other reasons for
  carrying out further research in the field covered by the project.
\end{formal}

\subsection{Host research expertise in the field (outgoing and return
  host)}

\begin{formal}
  The host institution must explain its level of experience on the
  research topic proposed and document its track record of work,
  including the main international collaborations. Information
  provided should include participation in projects, publications,
  patents and any other relevant results.
\end{formal}

\subsection{Quality of the group/supervisors (outgoing and return
  host)}

\begin{formal}
  Similar information as above should be provided for the scientist in
  charge of the supervision of the project. Where relevant, show that
  any gender issues associated to the proposal have been adequately
  taken into account. The host institution must demonstrate its track
  recordtraining achievements especially at an advanced level within
  the field of research.
\end{formal}


% --------------------------------------------------------------- Section
\section{Training  (maximum 2 pages)}

\subsection{Clarity and quality of the research training objectives
  for the researcher}

\begin{formal}
  State the training objectives and explain in detail how these can be
  beneficial for the (further) development of an independent research
  career.
\end{formal}

\subsection{Relevance and quality of additional research training as
  well as of transferable skills offered}

\begin{formal}
  Explain how the training provided will contribute to
  diversifying/broadening the competencies of the researcher, and how
  this will influence the researcher's career development. Outline
  complementary training and skills expected during the execution of
  the project (such as research management, presentation skills,
  ethics, etc.).
\end{formal}

\subsection{Host expertise in training experienced researchers in the
  field and capacity to provide mentoring/tutoring (outgoing and
  return host)}

\begin{formal}
  Give a short outline of the host's (outgoing and return phase)
  expertise in mentoring or tutoring researchers.
\end{formal}


% --------------------------------------------------------------- Section
\section{Researcher (maximum 7 pages which includes a CV and a list of
  main achievements)}

\subsection{Research experience}

\begin{formal}
The applicant must present a comprehensive description of his/her research experience. A 
scientific/professional CV must be provided and should mention explicitly: 
\begin{itemize}
\item academic achievements 
\item list of other professional activities 
\item any other relevant information. 
\end{itemize}
\end{formal}

\subsection{Research results including patents, publications, teaching
  etc. taking into account the level of experience}

\begin{formal}
  Outline the major achievements of the researcher. These may also
  include results in the form of funded projects, publications,
  patents, reports, invited participation in conferences etc., taking
  into account the level of experience. To help the expert evaluators
  better understand the level of skills and experience it is advisable
  to write a short description (around 250 words) of the major
  accomplishments mentioning the purpose, results, skills acquired,
  derived applications etc.
\end{formal}

\subsection{Independent thinking and leadership qualities}

\begin{formal}
  Describe the activities that reflect initiative, independent
  thinking, project management skills and leadership. Describe the
  potential that the researcher has for increasing and reinforcing
  these qualities.
\end{formal}

\subsection{Match between the fellow's profile and project}

\begin{formal}
  Show that the applicant's skills and experience are suitable for the
  project proposed.
\end{formal}

\subsection{Potential for reaching a position of professional maturity}

\begin{formal}
  Describe the potential of the researcher to reach professional
  maturity.
\end{formal}

\subsection{Potential to acquire new knowledge}

\begin{formal}
  Describe the researcher's ability to acquire new knowledge and skills.
\end{formal}

% --------------------------------------------------------------- Section
\section{Implementation (maximum 6 pages)}

\subsection{Quality of infrastructure/facilities and international
  collaborations of host (outgoing and return host)}

\begin{formal}
  The proposal must explain the level of experience of the outgoing
  host institution on the research topic proposed, including all
  international collaborations (outgoing and return host). Information
  provided should include participation in projects, publications,
  patents and any other relevant results. The facilities available in
  both the outgoing and returning host and their adequacy to the
  research project should be mentioned.
 
  Information on the capacity to provide training in complementary
  skills that can further aid the fellow in the reintegration period
  should be included. The host needs to specify what are the
  infrastructures available and whether these can respond to the needs
  set by the execution of the project.
 
  The European return host's qualities and capabilities to absorb and
  make use of the experience gained by the returning fellow should be
  described.
\end{formal}

\subsection{Practical arrangements for the implementation and
  management of the research project (outgoing and return host)}

\begin{formal}
  The applicant and the host institutions for both phases (outgoing
  and return) must be able to provide information on how the
  implementation and management of the fellowship will be
  achieved. The expert evaluators will be examine the practical
  arrangements that can have an impact on the feasibility and
  credibility of the project.
\end{formal}

\subsection{Feasibility and credibility of the project, including work plan}

\begin{formal}
  Provide a work plan that includes the goals that can help assess the
  progress of the project.  Mention the arrangements made in terms of
  supporting the reintegration phase of the fellow providing a career
  development plan where applicable. Where appropriate, describe the
  approach to be taken regarding the intellectual property that may
  arise from the research project.
\end{formal}

\subsection{Practical and administrative arrangements, and support for
  the hosting of the fellow (outgoing and return host)}

\begin{formal}
  The host of the outgoing phase should describe what practical
  arrangements are in place to host a researcher coming from another
  country. What support will be given to him/her to settle into their
  new host country (in terms of language teaching, help with local
  administration, obtaining of permits, accommodation, schools,
  childcare etc.) The host of the return phase should explain which
  measures are planned for the successful re-integration of the
  researcher.
\end{formal}

% --------------------------------------------------------------- Section
\section{Impact (maximum 4 pages)}

\subsection{Potential for acquiring competencies during the fellowship
  to improve the prospects of reaching and/or reinforcing a position
  of professional maturity, diversity and independence, in particular
  through exposure to transferable skills training}

\begin{formal}
  Describe the fellow's potential for acquiring (complementary)
  competencies and skills during the fellowship and what impact this
  will have on the prospects of reaching and/or reinforcing a position
  of professional maturity and/or research independence. Explain how
  the newly acquired skills and knowledge will be transferred from the
  Other Third Country to the return host.
\end{formal}

\subsection{Contribution to career development or re-establishment
  where relevant}

\begin{formal}
  How will the fellowship contribute in the medium- and long-term to
  the development of the fellow’s career? In the case of a fellow
  returning to research, how will his/her re-establishment be helped
  by the fellowship?
\end{formal}

\subsection{Potential for creating long term collaborations and
  mutually beneficial co-operation between Europe and the Other Third
  Country}

\begin{formal}
  What is the likelihood of continuing the collaboration between the
  two hosts after the end of the fellowship?
\end{formal}

\subsection{Contribution to European excellence and European
  competitiveness}

\begin{formal}
  Describe the extent to which the project will increase the
  attractiveness of the European Research Area for researchers,
  increase ERA competitiveness and produce long-term synergies and/or
  structuring effects.
\end{formal}

\subsection{Benefit of the mobility to the European Research Area}

\begin{formal}
  Describe how the proposed mobility is genuine and therefore
  beneficial to the European Research Area. Genuine mobility is
  considered to allow the researcher to work in a significantly
  different geograworked before.
\end{formal}

\subsection{Impact of the proposed outreach activities}

\begin{formal}
  Describe the outreach activities of the proposal to be implemented
  by the researcher during the project duration (for examples, see box
  on Outreach Activities below).
\end{formal}

% --------------------------------------------------------------- Section
\section{Ethical issues  (no maximum pages)}

MAIN ETHICAL ISSUES THAT MUST BE ADDRESSED
\begin{itemize}
\item Informed consent
\item Human embryonic stem cells
\item Privacy and data protection
\item Use of human biological samples and data
\item Research on animals
\item Research in developing countries
\item Dual use
\end{itemize}

AREAS EXCLUDED FROM FUNDING
\begin{itemize}
\item Research activity aiming at human cloning for reproductive
  purposes.
\item Research activity intended to modify the genetic heritage of
  human beings which could make such changes heritable (Research
  related to cancer treatment of the gonads can be financed).
\item Research activities intended to create human embryos solely for
  the purpose of research or for the purpose of stem cell procurement,
  including by means of somatic cell nuclear transfer.
\end{itemize}

Include the Ethical issues table below. If you indicate YES to any
issue, please identify the pages in the proposal where this ethical
issue is described. Answering 'YES' to some of these boxes does not
automatically lead to an ethical review. It enables the independent
experts to decide if an ethical review is required. If you are sure
that none of the issues apply to your proposal, simply tick the YES
box in the last row.

\clearpage

\subsection{Ethical issues table}

(Note: Research involving activities marked with an asterisk * in the
left column in the table below will be referred automatically to
Ethical Review)

% redefine vertically centered type for tabularx lines
\renewcommand{\tabularxcolumn}[1]{>{\arraybackslash}m{#1}}

\vspace{0.5cm}
%\noindent
\begin{tabularx}{\linewidth}{ | c | X | c | c | }
\rowcolor{black} & {\centering\arraybackslash \color{white} \bf Research on Human Embryo/Foetus} & {\color{white} \bf Yes} & {\color{white} \bf Page} \\ \hline
 * & Does the proposed research involve human Embryos?                                                      & & \\ \hline
 * & Does the proposed research involve human Foetal Tissues/ Cells?                                        & & \\ \hline
 * & Does the proposed research involve human Embryonic Stem Cells (hESCs)?                                 & & \\ \hline
 * & Does the proposed research on human Embryonic Stem Cells involve cells in culture?                     & & \\ \hline
 * & Does the proposed research on Human Embryonic Stem Cells involve the derivation of cells from Embryos? & & \\ \hline
   & I CONFIRM THAT NONE OF THE ABOVE ISSUES APPLY TO MY PROPOSAL                                           & & \cellcolor[gray]{0.8}\\ \hline 
\end{tabularx}

\vspace{0.5cm}
%\noindent
\begin{tabularx}{\linewidth}{ | c | X | c | c | }
\rowcolor{black} & {\centering\arraybackslash \color{white} \bf Research on Humans} & {\color{white} \bf Yes} & {\color{white} \bf Page} \\ \hline
 * & Does the proposed research involve children?                         & & \\ \hline
 * & Does the proposed research involve patients?                         & & \\ \hline
 * & Does the proposed research involve persons not able to give consent? & & \\ \hline
 * & Does the proposed research involve adult healthy volunteers?         & & \\ \hline
   & Does the proposed research involve Human genetic material?           & & \\ \hline
   & Does the proposed research involve Human biological samples?         & & \\ \hline
   & Does the proposed research involve Human data collection?            & & \\ \hline
   & I CONFIRM THAT NONE OF THE ABOVE ISSUES APPLY TO MY PROPOSAL         & & \cellcolor[gray]{0.8}\\ \hline 
\end{tabularx}

\vspace{0.5cm}
%\noindent
\begin{tabularx}{\linewidth}{ | c | X | c | c | }
\rowcolor{black} & {\centering\arraybackslash \color{white} \bf Privacy} & {\color{white} \bf Yes} & {\color{white} \bf Page} \\ \hline
   & Does the proposed research involve processing of genetic information or personal data (e.g. health, sexual lifestyle, ethnicity, political opinion, religious or philosophical conviction)? & & \\ \hline
   & Does the proposed research involve tracking the location or observation of people? & & \\ \hline
   & I CONFIRM THAT NONE OF THE ABOVE ISSUES APPLY TO MY PROPOSAL & & \cellcolor[gray]{0.8}\\ \hline
\end{tabularx}

\vspace{0.5cm}
%\noindent
\begin{tabularx}{\linewidth}{ | c | X | c | c | }
\rowcolor{black} & {\centering\arraybackslash \color{white} \bf Research on Animals} & {\color{white} \bf Yes} & {\color{white} \bf Page} \\ \hline
   & Does the proposed research involve research on animals?      & & \\ \hline
   & Are those animals transgenic small laboratory animals?       & & \\ \hline
   & Are those animals transgenic farm animals?                   & & \\ \hline
 * & Are those animals non-human primates?                        & & \\ \hline
   & Are those animals cloned farm animals?                       & & \\ \hline
   & I CONFIRM THAT NONE OF THE ABOVE ISSUES APPLY TO MY PROPOSAL & & \cellcolor[gray]{0.8}\\ \hline
\end{tabularx}

\vspace{0.5cm}
%\noindent
\begin{tabularx}{\linewidth}{ | c | X | c | c | }
\rowcolor{black} & {\centering\arraybackslash \color{white} \bf Research Involving Developing Countries} & {\color{white} \bf Yes} & {\color{white} \bf Page} \\ \hline
   & Does the proposed research involve the use of local resources (genetic, animal, plant, etc)?                             & & \\ \hline
   & Is the proposed research of benefit to local communities (e.g. capacity building, access to healthcare, education, etc)? & & \\ \hline
   & I CONFIRM THAT NONE OF THE ABOVE ISSUES APPLY TO MY PROPOSAL                                                             & & \cellcolor[gray]{0.8}\\ \hline
\end{tabularx}

\vspace{0.5cm}
%\noindent
\begin{tabularx}{\linewidth}{ | c | X | c | c | }
\rowcolor{black} & {\centering\arraybackslash \color{white} \bf Dual Use} & {\color{white} \bf Yes} & {\color{white} \bf Page} \\ \hline
   & Research having direct military use                          & & \\ \hline
   & Research having the potential for terrorist abuse            & & \\ \hline
   & I CONFIRM THAT NONE OF THE ABOVE ISSUES APPLY TO MY PROPOSAL & & \cellcolor[gray]{0.8}\\ \hline
\end{tabularx}

% -----------------------------------------------------------------------
% End page

\clearpage

\thispagestyle{empty}
\begin{center}

\Huge{\bf ENDPAGE}

\vspace{4cm}

\Large{PEOPLE\\MARIE CURIE ACTIONS}

\vspace{4cm}

\Large{\bf \CallName}

\Large{\bf Call: \CallID}

\vspace{4cm}

\huge{PART B}

\vspace{4cm}

\huge{"\AppShortTitle"}

\end{center}

\end{document}
